\documentclass[openany]{book}
\usepackage{lrcsstyle}

\title{\textbf{LRCS Press Style}}
\author{Author Name}
\affil{\href{mailto:example@lrcs.ac}{example@lrcs.ac}\\
\bigskip
\textit{Institution Line 1}\\
\textit{Institution Line 2}}
\date{
\vspace{1cm}
\includegraphics[width=0.33\textwidth]{cover_pic.png}
}

\begin{document}

\maketitle

\chapter*{Preface}
\addcontentsline{toc}{chapter}{Preface}

\lipsum[1-4]

\bigskip

\begin{flushright}
    Author Name\\
    \today
\end{flushright}


\tableofcontents


\chapter{Chapter-Example}

\section{Section-Example}

\lipsum

\section{Section-Example}

\subsection{Subsection-Example}

\lipsum

\subsection{Subsection-Example}

\lipsum

\chapter{Chapter-Example}

\section{Equations}

You can typeset equations as follow:
\begin{equation}
    x^2 \frac{d^2y}{dx^2} + x \frac{dy}{dx} + (x^2 - \alpha^2)y = 0. \label{eq:some-eq}
\end{equation}

You can reference your equation using \Cref{eq:some-eq}.

\section{Tables}

You can insert tables as follow:

\begin{table}[h]
    \centering
    \caption{A Fancy Title}
    \begin{tblr}{width=0.75\linewidth,colspec={|[1pt]c|c|c|[1pt]},colsep=14pt,rowspec={|[1pt]Q|[1pt]Q|Q|Q|[1pt]},rowsep=4pt}
        Something & Here & Ummm... \\
        Maybe & There & ? \\
        This & Time & Here! \\
        Sorry & Finally & There
    \end{tblr}
    \label{tab:some-lable}
\end{table}

You can reference your table using \Cref{tab:some-lable}.

\end{document}
